\documentclass{article}
\usepackage{mathptmx,amssymb,amsmath,amscd,enumerate}
\setlength{\textwidth}{16.5cm}
\setlength{\oddsidemargin}{-0.1cm}
\setlength{\evensidemargin}{-0.1cm}
\setlength{\textheight}{23cm}
\setlength{\topmargin}{-1.3cm}

\def\ge{\geqslant}
\def\le{\leqslant}
\def\phi{\varphi}
\def\to{\longrightarrow}
\def\mapsto{\longmapsto}
\def\la{\langle}
\def\ra{\rangle}
\def\lcm{\operatorname{lcm}}
\renewcommand{\mod}{\,\operatorname{mod}\,}
\renewcommand{\arraystretch}{1.5}

\def\bar{\overline}
\def\image{\operatorname{Image}}
\def\Perm{\operatorname{Perm}}

\def\CC{{\mathbb C}}
\def\FF{{\mathbb F}}
\def\NN{{\mathbb N}}
\def\QQ{{\mathbb Q}}
\def\RR{{\mathbb R}}
\def\ZZ{{\mathbb Z}}

\pagestyle{empty}

\begin{document}
\noindent\textbf{Math 5405, Spring 2017, Assignment 3} \hfill \textbf{Due in class Tuesday, March 21}
\bigskip

\begin{enumerate}

%%%%%%%%%%%%%%%%%%%%%%%%%%%%%%%%%%%%%%%%%%%%%%%%%%%%%%
\item An RSA-encrypted message reads $8059008699$ where $e=113$ and the modulus $m=10993522499$ is a product of two primes that differ by $2$. Decrypt this, and express the final answer in terms of the nine letter alphabet from the lecture notes.

%%%%%%%%%%%%%%%%%%%%%%%%%%%%%%%%%%%%%%%%%%%%%%%%%%%%%%
\item Compute $255^2$ and $317^2$ modulo $64777$, and use these to factor $64777$.

%%%%%%%%%%%%%%%%%%%%%%%%%%%%%%%%%%%%%%%%%%%%%%%%%%%%%%
\item Using the quadratic sieve method, factor \ $n=46698343$. List only the congruences that you use to construct
\[
x^2\equiv y^2\mod n.
\]

%%%%%%%%%%%%%%%%%%%%%%%%%%%%%%%%%%%%%%%%%%%%%%%%%%%%%%
\item Determine if $561$ is a square modulo the prime $1151$ using (a) only Legendre symbols, and (b) Jacobi symbols.

%%%%%%%%%%%%%%%%%%%%%%%%%%%%%%%%%%%%%%%%%%%%%%%%%%%%%%
\item Use the Solovay-Strassen test to show that the following numbers are composite:
\[
\text{(a)}\ 899\quad\text{(b)}\ 1729\quad\text{(c)}\ 3599
\]

%%%%%%%%%%%%%%%%%%%%%%%%%%%%%%%%%%%%%%%%%%%%%%%%%%%%%%
\item We saw how to solve $x^2\equiv a\mod p$ when $p\equiv3\mod 4$; here is an analogue for $p\equiv5\mod 8$:

Suppose $a$ is a quadratic residue modulo a prime $p$ with $p\equiv 5\mod 8$.
\begin{enumerate}
\item Show that $a^{(p-1)/4}\equiv\pm1\mod p$.
\item If $a^{(p-1)/4}\equiv1\mod p$, prove that the solutions of $x^2\equiv a\mod p$ are $\pm a^{(p+3)/8}$.
\item If $a^{(p-1)/4}\equiv-1\mod p$, prove that the solutions of $x^2\equiv a\mod p$ are $\pm 2a (4a)^{(p-5)/8}$.
\end{enumerate}

%%%%%%%%%%%%%%%%%%%%%%%%%%%%%%%%%%%%%%%%%%%%%%%%%%%%%%
\item Consider the prime $p=5643653$. Using the previous problem, solve the following in $\ZZ/p$.
\[
\text{(a)}\ x^2=435645\qquad\text{(b)}\ x^2=1892117
\]

%%%%%%%%%%%%%%%%%%%%%%%%%%%%%%%%%%%%%%%%%%%%%%%%%%%%%%
\item Compute $(1,5)+(9,3)$ on the elliptic curve $y^2=x^3+2x+3\mod 19$.

%%%%%%%%%%%%%%%%%%%%%%%%%%%%%%%%%%%%%%%%%%%%%%%%%%%%%%
\item Complete the addition table below where $P=(-1,0)$, \ $Q=(0,1)$, and $R=(2,3)$, are points on the elliptic curve $y^2=x^3+1$. (This shows that these six points form a subgroup.)

\[
\begin{tabular}{|r||c|c|c|c|c|c|}
\hline
$+$ & $\infty$ & $P$ & $Q$ & $-Q$ & $R$ & $-R$\\
\hline
\hline
$\infty$ & \phantom{$-Q$} & \phantom{$-Q$} & \phantom{$-Q$} & \phantom{$-Q$} & \phantom{$-Q$} & \\
\hline
$P$ & & & & & & \\
\hline
$Q$ & & & & & & \\
\hline
$-Q$ & & & & &  & \\
\hline
$R$ & & & & & & \\
\hline
$-R$ & & & &  & & \\
\hline
\end{tabular}
\]

%%%%%%%%%%%%%%%%%%%%%%%%%%%%%%%%%%%%%%%%%%%%%%%%%%%%%%
\item Determine the number of elements in the elliptic curve group of $y^2=x^3+8\mod 19$.

%%%%%%%%%%%%%%%%%%%%%%%%%%%%%%%%%%%%%%%%%%%%%%%%%%%%%%

\end{enumerate}
\end{document}
